Das Ziel der vorliegenden Seminararbeit ist es, einen generellen Überblick über das Thema der Fragmentierung von 3D-Objekten zu geben, indem verschiedene Algorithmen
beschrieben werden. Um diesen Überblick zu geben, wurde eine ausführliche Literaturrecherche durchgeführt. Uas dieser Recherche ist hervorgekommen, dass es 
grundsätzlich zwei Ansätze gibt eine Frakturierung durchzuführen, nicht physikalisch und physikalisch. Nicht physikalische Ansätze sind meist schneller zu berechnen,
jedoch ist die Qualität der Fragmentierung des 3D-Objektes niedriger als die der physikalischen Algorithmen. Deshalb werden physikalische Ansätze oft bei Simulationen verwendet
und nicht physikalische Methoden bei beispielsweise Videospielen.