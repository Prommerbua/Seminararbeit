%
% FH Technikum Wien
% !TEX encoding = UTF-8 Unicode
%
% Erstellung von Master- und Bachelorarbeiten an der FH Technikum Wien mit Hilfe von LaTeX und der Klasse TWBOOK
%
% Um ein eigenes Dokument zu erstellen, müssen Sie folgendes ergänzen:
% 1) Mit \documentclass[..] einstellen: Master- oder Bachelorarbeit, Studiengang und Sprache
% 2) Mit \newcommand{\FHTWCitationType}.. Zitierstandard festlegen (wird in der Regel vom Studiengang vorgegeben - bitte erfragen)
% 3) Deckblatt, Kurzfassung, etc. ausfüllen
% 4) und die Arbeit schreiben (die verwendeten Literaturquellen in Literatur.bib eintragen)
%
% Getestet mit TeXstudio mit Zeichenkodierung ISO-8859-1 (=ansinew/latin1) und MikTex unter Windows
% Zu beachten ist, dass die Kodierung der Datei mit der Kodierung des paketes inputenc zusammen passt!
% Die Kodierung der Datei twbook.cls MUSS ANSI betragen!
% Bei der Verwendung von UTF8 muss dnicht nur die Kodierung des Dokuments auf UTF8 gestellt sein, sondern auch die des BibTex-Files!
%
% Bugreports und Feedback bitte per E-Mail an latex@technikum-wien.at 
%
% Versionen
% *) V0.7: 9.1.2015, RO: Modeline angepasst und verschoben
% *) V0.6: 10.10.2014, RO: Weitere Anpassung an die UK
% *) V0.5: 8.8.2014, WK: Literaturquellen überarbeitet und angepasst
% *) V0.4: 4.8.2014, WK: Initalversion in SVN eingespielt
%
\documentclass[MGS,Seminar,ngerman]{twbook}%\documentclass[Bachelor,BMR,ngerman]{twbook}
\usepackage[utf8]{inputenc}
\usepackage[T1]{fontenc}

%
% Bitte in der folgenden Zeile den Zitierstandard festlegen
\newcommand{\FHTWCitationType}{HARVARD} % IEEE oder HARVARD möglich - wenn Sie zwischen IEEE und HARVARD wechseln, bitte die temorären Dateien (aux, bbl, ...) löschen
%
\ifthenelse{\equal{\FHTWCitationType}{HARVARD}}{\usepackage{harvard}}{\usepackage{bibgerm}}

% Definition Code-Listings Formatierung:
\usepackage[final]{listings}
\lstset{captionpos=b, numberbychapter=false,caption=\lstname,frame=single, numbers=left, stepnumber=1, numbersep=2pt, xleftmargin=15pt, framexleftmargin=15pt, numberstyle=\tiny, tabsize=3, columns=fixed, basicstyle={\fontfamily{pcr}\selectfont\footnotesize}, keywordstyle=\bfseries, commentstyle={\color[gray]{0.33}\itshape}, stringstyle=\color[gray]{0.25}, breaklines, breakatwhitespace, breakautoindent}
\lstloadlanguages{[ANSI]C, C++, [gnu]make, gnuplot, Matlab}

%Formatieren des Quellcodeverzeichnisses
\makeatletter
% Setzen der Bezeichnungen für das Quellcodeverzeichnis/Abkürzungsverzeichnis in Abhängigkeit von der eingestellten Sprache
\providecommand\listacroname{}
\@ifclasswith{twbook}{english}
{%
    \renewcommand\lstlistingname{Code}
    \renewcommand\lstlistlistingname{List of Code}
    \renewcommand\listacroname{List of Abbreviations}
}{%
    \renewcommand\lstlistingname{Quellcode}
    \renewcommand\lstlistlistingname{Quellcodeverzeichnis}
    \renewcommand\listacroname{Abkürzungsverzeichnis}
}
% Wenn die Option listof=entryprefix gewählt wurde, Definition des Entyprefixes für das Quellcodeverzeichnis. Definition des Macros listoflolentryname analog zu listoflofentryname und listoflotentryname der KOMA-Klasse
\@ifclasswith{scrbook}{listof=entryprefix}
{%
    \newcommand\listoflolentryname\lstlistingname
}{%
}
\makeatother
\newcommand{\listofcode}{\phantomsection\lstlistoflistings}

% Die nachfolgenden Pakete stellen sonst nicht benötigte Features zur Verfügung
\usepackage{blindtext}

%
% Einträge für Deckblatt, Kurzfassung, etc.
%
\title{Arbeitstitel\\Arbeitstitel}
\author{Michael Prommer, BSc}
\studentnumber{XXXXXXXXXXXXXXX}
%\author{Titel Vorname Name, Titel\and{}Titel Vorname Name, Titel}
%\studentnumber{XXXXXXXXXXXXXXX\and{}XXXXXXXXXXXXXXX}
\supervisor{Titel Vorname Name, Titel}
%\supervisor[Begutachter]{Titel Vorname Name, Titel}
%\supervisor[Begutachterin]{Titel Vorname Name, Titel}
%\secondsupervisor{Titel Vorname Name, Titel}
%\secondsupervisor[Begutachter]{Titel Vorname Name, Titel}
%\secondsupervisor[Begutachterinnen]{Titel Vorname Name, Titel}
\place{Wien}
\kurzfassung{\blindtext}
\schlagworte{Schlagwort1, Schlagwort2, Schlagwort3, Schlagwort4}
\outline{\blindtext}
\keywords{Keyword1, Keyword2, Keyword3, Keyword4}
%\acknowledgements{\blindtext}

\begin{document}

%Festlegungen für den HARVARD-Zitierstandard
\ifthenelse{\equal{\FHTWCitationType}{HARVARD}}{
\bibliographystyle{Harvard_FHTW_MR.bst}%Zitierstandard FH Technikum Wien, Studiengang Game Engineering
\citationstyle{dcu}%Correct citation-style (Harvardand, ";" between citations, "," between author and year)
\citationmode{abbr}%use "et al." with first citation
\iflanguage{ngerman}{
    %Deutsch Neue Rechtschreibung
    \newcommand{\citepic}[1]{(Quelle: \protect\cite{#1})}%Zitat: Bild
    \newcommand{\citefig}[2]{(Quelle: \protect\cite{#1}, S. #2)}%Zitat: Bild aus Dokument
    \newcommand{\citefigm}[2]{(Quelle: modifiziert "ubernommen aus \protect\cite{#1}, S. #2)}%Zitat: modifiziertes Bild aus Dokument
    \newcommand{\citep}{\citeasnoun}%In-Line Zitiat entweder mit \citep{} oder \citeasnoun{}
    \newcommand{\acessedthrough}{Verf{\"u}gbar unter:}%Für URL-Angabe
    \newcommand{\acessedthroughp}{Verf{\"u}gbar bei:}%Für URL-Angabe (Geschützte Datenbank, Zugriff durch FH)
    \newcommand{\acessedat}{Zugang am}%Für URL-Datum-Angabe
    \newcommand{\singlepage}{S.}%Für Seitenangabe (einzelne Seite)
    \newcommand{\multiplepages}{S.}%Für Seitenangabe (mehrere Seiten)
    \newcommand{\chapternr}{K.}%Für Kapitelangabe
    \renewcommand{\harvardand}{\&}%Harvardand in Zitaten
    \newcommand{\abstractonly}{ausschließlich Abstract}
    \newcommand{\edition}{. Auflage}%Angabe der Auflage
}{
\iflanguage{german}{
    %Deutsch
    \newcommand{\citepic}[1]{(Quelle: \protect\cite{#1})}%Zitat: Bild
    \newcommand{\citefig}[2]{(Quelle: \protect\cite{#1}, S. #2)}%Zitat: Bild aus Dokument
    \newcommand{\citefigm}[2]{(Quelle: modifiziert "ubernommen aus \protect\cite{#1}, S. #2)}%Zitat: modifiziertes Bild aus Dokument
    \newcommand{\citep}{\citeasnoun}%In-Line Zitiat entweder mit \citep{} oder \citeasnoun{}
    \newcommand{\acessedthrough}{Verf{\"u}gbar unter:}%Für URL-Angabe
    \newcommand{\acessedthroughp}{Verf{\"u}gbar bei:}%Für URL-Angabe (Geschützte Datenbank, Zugriff durch FH)
    \newcommand{\acessedat}{Zugang am}%Für URL-Datum-Angabe
    \newcommand{\singlepage}{S.}%Für Seitenangabe (einzelne Seite)
    \newcommand{\multiplepages}{S.}%Für Seitenangabe (mehrere Seiten)
    \newcommand{\chapternr}{K.}%Für Kapitelangabe
    \renewcommand{\harvardand}{\&}%Harvardand in Zitaten
    \newcommand{\abstractonly}{ausschließlich Abstract}
    \newcommand{\edition}{. Auflage}%Angabe der Auflage
}{
    %Englisch
    \newcommand{\citepic}[1]{(Source: \protect\cite{#1})}%Zitat: Bild
    \newcommand{\citefig}[2]{(Source: \protect\cite{#1}, p. #2)}%Zitat: Bild aus Dokument
    \newcommand{\citefigm}[2]{(Source: taken with modification from \protect\cite{#1}, p. #2)}%Zitat: modifiziertes Bild aus Dokument
    \newcommand{\citep}{\citeasnoun}%In-Line Zitiat entweder mit \citep{} oder \citeasnoun{}
    \newcommand{\acessedthrough}{Available at:}%Für URL-Angabe
    \newcommand{\acessedthroughp}{Available through:}%Für URL-Angabe (Geschützte Datenbank, Zugriff durch FH)
    \newcommand{\acessedat}{Accessed}%Für URL-Datum-Angabe
    \newcommand{\singlepage}{p.}%Für Seitenangabe (einzelne Seite)
    \newcommand{\multiplepages}{pp.}%Für Seitenangabe (mehrere Seiten)
    \newcommand{\chapternr}{Ch.}%Für Kapitelangabe
    \renewcommand{\harvardand}{\&}%Harvardand in Zitaten
    \newcommand{\abstractonly}{Abstract only}
    \newcommand{\edition}{~edition}%Edition -> note, that you have to write "edition = {2nd},"!
}}}

\maketitle
%
% .. und hier beginnt die eigentliche Arbeit. Viel Erfolg beim Verfassen!
%
\chapter{Erste Überschrift der Ebene 1 (chapter)}

\chapter{Einführung}

In modernen Videospielen und Filmen werden verschiedenste Effekte verwendet um dem Spieler die Welt glaubhafter und realer wirken zu lassen.
Einer davon ist die Zerstörung von unterschiedlichen Objekten im Spiel wie einfache Gegenstände oder in manchen Spielen sogar zerstörbare Landschaften.
Aufgrund des hohen Rechenaufwands wird in Spielen jedoch meist ein statischer Ansatz für solch destruktive Objekte verwendet, das heißt ein Model wird von einem Artist 
bereits vorher fragmentiert modelliert und zum gewünschten Zeitpunkt im Spiel einfach ausgetauscht. Dieser Ansatz hat jedoch auch Nachteile. 
Zum Beispiel wird die Entwicklungszeit eines Spieles verlängert, da der Artist 3D-Modelle mit unterschiedlichen Frakturierungen erstellen muss,
und zusätzlich kann es sein, dass die Spielerinteraktion in der Spielumgebung durch vorgebrochene Objekte einteschränkt wird \cite{Najim.DynamicFracturing}.
Außerdem hat diese Methode den Nachteil, dass das Muster der Fragmentierung nicht mit der Einschlagstelle übereinstimmt und dass die Anzahl der 
hierarchischen Burchstufen bereits festgelegt ist \cite{Mueller.RealTimeDynamicFractureVACD}.

Durch bisherige Bemühungen auf diesem Gebiet konnten Algorithmen entwickelt werden, welche in Echtzeit ausgeführt werden können und dadurch
einen dynamischeren Effekt erzielen. Dadurch können beispielsweise interessantere und nicht repetitive Fragmentierungsmuster erstellt werden und außerdem fällt die 
Entwicklungszeit der 3D-Modellen mit unterschiedlichen Frakturierungen weg.
Diese Algorithmen zur Erstellung von Fragmentierungen basieren beispielsweise auf dem Voronoi Diagramm.

\section{Erste Überschrift Tiefe 2 (section)}

\subsection{Erste Überschrift Tiefe 3 (subsection)}

\subsubsection{Erste Überschrift Tiefe 4 (subsubsection)}

\chapter{Zweite Überschrift der Tiefe 1 (chapter)}

\section{Zweite Überschrift Tiefe 2 (section)}

\section{Zweite Überschrift Tiefe 2 (section)}

\subsection{Zweite Überschrift Tiefe 3 (subsection)}

\subsection{Dritte Überschrift Tiefe 3 (subsection)}

\subsubsection{Zweite Überschrift Tiefe 4 (subsubsection)}




%
% Hier beginnen die Verzeichnisse.
%
\clearpage
\ifthenelse{\equal{\FHTWCitationType}{HARVARD}}{}{\bibliographystyle{gerabbrv}}
\bibliography{Literatur}
\clearpage

% Das Abbildungsverzeichnis
% \listoffigures
% \clearpage

% Das Tabellenverzeichnis
% \listoftables
% \clearpage

% Das Quellcodeverzeichnis
% \listofcode
% \clearpage

% \phantomsection
% \addcontentsline{toc}{chapter}{\listacroname}
% \chapter*{\listacroname}
% \begin{acronym}[XXXXX]
%     \acro{ABC}[ABC]{Alphabet}
%     \acro{WWW}[WWW]{world wide web}
%     \acro{ROFL}[ROFL]{Rolling on floor laughing}
% \end{acronym}

\end{document}