\chapter{Fazit}

Zusammenfassend lässt sich sagen, dass es einige Möglichkeiten gibt, Fragmentierungen an Objekten vorzunehmen. Ob nun ein nicht physikalisch basierter oder
physikalisch basierter Ansatz für ein Problem verwendet wird, ist abhängig von der gewünschten Realität der Zerstörung und der benötigten Rechenzeit.
Für die genaue Simulation einer Zerstörung, wo die Rechenzeit keine Rolle spielt, würde eine physikalische Methode wesentlich bessere Ergebnisse erzielen.
Sollte die Fragmentierung eher schnell passieren und abhängig von der Interaktion eines Benutzers, werden die nicht physikalischen Methoden durschnittlich
besser abschneiden, im Hinblick auf Performance.  

Es konnte hier keine endgültige Antwort gegeben werden, welche Methode oder Algorithmus des jeweiligen Ansatzes am besten ist, da der Fokus dieser Arbeit
anders gesetzt ist, und nur einen Überblick über die Methoden gibt.

Dadurch ergibt sich aber eine Fragstellung, die noch weitere Untersuchungen erfordert, nämlich die Algorithmen generell im Hinblick auf Performance, Speicherverbrauch 
und Qualität der Fragmentierung zu vergleichen.
%Grenzen der Arbeit
%keine endgültige Antwort gegeben werden, welche Methode oder Algorithmus am besten ist, da Fokus anders gesetzt ist
%es konnte hier nur am rande besprochen werden...


%Ausblick(Genauer Performance Unterschied zwischen Methoden)
% Hier empfiehlt sich eine weitere Untersuchung, denn…

% Eine Fragestellung, die noch weiterer Untersuchungen erfordert, ist…
